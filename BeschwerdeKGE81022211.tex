\documentclass[paper=a4, onesite]{scrreprt}

\usepackage[ngerman]{babel}

\usepackage{acronym}

\usepackage{pdfpages}

\usepackage{enumitem}

\usepackage{geometry}
\geometry{a4paper, top=2cm, bottom=2cm}




\makeatletter
\renewcommand\chapter{
	\thispagestyle{\chapterpagestyle}
	\global\@topnum\z@
	\@afterindentfalse
	\secdef\@chapter\@schapter
}
\makeatother


\renewcommand*\chapterheadendvskip{%
	\vspace*{1\baselineskip plus .1\baselineskip minus .167\baselineskip}}

\renewcommand*\familydefault{\sfdefault}


\setlength{\parindent}{0em}


\newcounter{rz}
\newcommand{\Rz}{\addtocounter{rz}{1}\marginpar{\texttt{(\textit{\arabic{rz}})}}}

% \usepackage{hyperref}


\usepackage{csquotes}

\usepackage[backend=biber,style=alphabetic,urldate=iso8601,date=iso8601]{biblatex}
\addbibresource{BeschwerdeKGE81022211.bib}

\newcounter{rtaskno}
\newcommand{\rtask}[1]{\label{#1}}


\usepackage{scrletter}
\setkomavar{fromaddress}{Jens Fischer\\Ziegelweg 4\\4102 Binningen}
\setkomavar{backaddress}{Jens Fischer\\Ziegelweg 4\\4102 Binningen}
\setkomavar{subject}{\\Betreff: Beschwerde bzgl. Urteil des  Kantonsgericht BL
	\vspace{1em}\\
	Aufhebung des Urteils des Kantonsgerichts Basel-Landschaft, Abteilung Verfassungs- und Verwaltungsrecht, vom 11. Oktober 2022 (KGE 810 22 211)\\
	 
}

\setlength{\parindent}{0em}

\begin{document}
	% If you want headings on subsequent pages,
	% remove the ``%'' on the next line:
	% \pagestyle{headings}
	
	
	\begin{letter}{Schweizerisches Bundesgericht\\II. öffentlich-rechtliche Abteilung\\Av. du Tribunal-Fédéral 29\\1000 Lausanne 14}
		
		\opening{Sehr geehrte Damen und Herren,}
		
		ich habe am 07. Oktober 2022 beim \ac{KG} eine Beschwerde bzgl. der Verfügung vom 27. September 2022 \cite{Auszug2} eingereicht. Das \ac{KG} trat auf die Beschwerde mittels eines präsidentiellen Entscheids vom 11. Oktober 2022 \cite{KGE81022211} nicht ein. In seiner Erwägung führt die präsidierende Richterin Frau Franziska Preiswerk-Vögtli aus, dass meine Beschwerdeschrift keine hinreichende Begründung enthalte und ausserdem missbräuchlich wäre. Das Missbräuchliche wird damit begründet, dass meine Beschwerdeführung ohne konkreten Vorteil und vernünftiges Ziel verfolgende Rechthaberei sei.\\
		
		Mit meinem Beschwerdeschreiben an das Bundesgericht werde ich aufzeigen, dass die Aussage des \ac{KG} auf mehrfachen Verstosses gegen das rechtliche Gehör, u.a. werden rechtserhebliche Tatsache aus der Verfügung vom 27. Oktober 2022 \cite{Auszug2} und meiner Beschwerdeschrift vom 07. Oktober 2022 \cite{BSAuszug2} nicht gehört, sowie dem Feststellen eines unwahren Sachverhalts beruht. Wobei sich der Sachverhalt des Urteils nur aus den aufgeführten Erwägungen ergibt, denn im Urteil ist kein eigener Abschnitt bzgl. des rechtserheblichen Sachverhalts vorhanden.\\
		
		
		Hochachtungsvoll\vspace{1.5em}
		
		Jens Fischer\\

\end{letter}

\tableofcontents
\newpage 
\chapter{Rechtsbegehren}
\begin{enumerate}
	\item Es sei das Urteil des \acl{KG}, vom 11. Oktober 2022 (810 22 211) \cite{KGE81022211} vollumfänglich aufzuheben.
	\item Es sei das Verfahren zur Neubeurteilung an die Vorinstanz zurückzuweisen.
	\item Unter o/e Kostenfolge.
\end{enumerate}	

\chapter{Verfahrensantrag}
\Rz Es seinen die Akten des Verfahrens vor dem \acl{KG} mit der Verfahrensnummer 810 22 211 beizuziehen.

\chapter{Formelles}

\section{Anfechtungsobjekt}
\Rz Beim angefochtenen Urteil KGE 810 22 211 vom 11. Oktober 2022 \cite{KGE81022211} handelt es sich um einen Endentscheid des \acl{KG} betreffend kantonaler Bewilligung im Einzelfall der "`\textit{\acl{SFaP}}"' in Form einer Verfügung. Folglich betrifft das vorgenannte Urteil eine Angelegenheit des öffentlichen Rechts, ohne dass eine der im Gesetz genannten Ausnahmen gegeben ist. Damit stellt es ein zulässiges Anfechtungsobjekt dar (Art. 82 lit. a BGG; Art. 90 BGG).\\

\section{Vorinstanz}
\Rz \rtask{Vorinstanz}Das Kantonsgericht Basel-Landschaft ist die letzte kantonale Instanz. Eine Beschwerde beim Bundesverwaltungsgericht ist nicht zulässig. Somit handelt es sich beim angefochtenen Urteil um einen Entscheid einer Vorinstanz gegen den nach Art. 86 Abs. 1 lit. d BGG eine Beschwerde zulässig ist.

\section{Beschwerdegründe}\label{bgruende}
\Rz In der vorliegenden Beschwerde wird die Verletzung von Bundesrecht (Art. 95 lit. a BGG) durch Verletzung des Verbots des rechtlichen Gehörs (Art. 29 Abs. 2 BV, Art. 29 Abs. 1 BV), des Verbots der Rechtsverweigerung im engeren Sinne (Art. 29 Abs. 1 BV), des Verbots des überspitzen Formalismus (Art. 29 Abs. 1 BV) sowie des Willkürverbots (Art. 9 BV) gerügt. Ferner wird auch die Verletzung von Art. 6 Ziff. 1 EMRK gerügt, womit der Beschwerdegrund der Verletzung von Völkerrecht (Art. 95 lit. b BGG) einschlägig ist. Überdies erfolgt auch die Rüge der unrichtigen Feststellung des Sachverhalts (Art. 97 BGG).\\

\section{Beschwerdelegitimation}

\subsection{Parteistellung}
\Rz Ich bin der gesetzliche Vertreter von Julian Fischer (geb. 21. September 2007). Im Beschwerdeverfahren geht es um die Bewilligung einer "`\acl{SFaP}"', bei der es im Grunde um die Kostenübernahme des Kantons Basel-Landschaft im Einzelfall für einen Privatschulbesuch handelt. Zusätzlich werde ich  im bescherten Urteil \cite{KGE81022211} als Verfahrensbeteiligter (Beschwerdeführer) aufgeführt. Somit dürfte unbestritten sein, dass mir eine Parteistellung im von mir mit diesem Schreiben beantragten Beschwerdeverfahren zukommt.

\subsection{Schutzwürdiges Interesse}
\Rz In Beschwerdeverfahren bzgl. der \textit{\acl{SFaP}} (\ac{SFaP}) geht es im Allgemeinen um die Frage, ob die Bewilligung der \ac{SFaP} zu Unrecht verweigert wurde.\\ 

\Rz In diesem Beschwerdeverfahren ist jedoch die spezielle Situation, dass gerade Einigkeit darüber herrscht, dass die \ac{SFaP} für unseren Sohn bis Ende der obligatorischen Schulzeit benötigt und diese am \ac{BZB} stattfinden soll.\\


\Rz Mein schutzwürdiges Interesse liegt daher darin begründet, dass ich eine für mich nutzbare Bewilligung bzgl. der \ac{SFaP} erhalte. Weshalb die mir vorliegende Bewilligung der \ac{SFaP} für mich nicht nutzbar ist, wird im Abschnitt \ref{Rechtliches} aufgezeigt.



\section{Frist}
\Rz Das beschwerte Urteil des \acl{KG} vom 11. Oktober 2022 wurde mir am 17. Oktober 2022 per Einschreiben zugestellt (siehe  \cite{CouvertKG}). Damit beginnt die 30 tägige Beschwerdefrist Dienstag den 18. Oktober 2022 und endet mit dem 17. November 2022 (Poststempel der Schweizer Post). Das Dossier mit meiner Beschwerde wurde vor Ablauf der Beschwerdefrist bei der Post als Einschreiben aufgebeben, womit die Beschwerde fristgerecht beim \ac{BuGe} eingereicht ist.

\section{Form und Inhalt}
\Rz Die vorliegende Beschwerde erfolgt schriftlich und ist in einer Amtssprache abgefasst. Ferner bezeichnet sie den angefochtenen Akt, enthält die Rechtsbegehren und deren Begründung mit Angabe der Beweismittel sowie die Unterschrift des Rechtsvertreters des Beschwerdeführers (Art. 42 Abs. 1 BGG). In der Begründung in den RZ 24 ff. wird in gedrängter Form dargelegt, inwiefern der angefochtene Entscheid Recht verletzt, womit den Voraussetzungen von Art. 42 Abs. 2 BGG entsprochen wird.\\

\Rz Mittels vorliegender Beschwerde wird explizit die Verletzung von Grundrechten gerügt (vgl. \ref{bgruende} Beschwerdegründe). Insofern wird dem Rügeprinzip nach Art. 106 Abs. 2 BGG entsprochen.


\section{Rechtliches bzgl. der Verfügung vom 27. Oktober 2022}\label{RechtAuszug2}
\Rz Im Zuge meines Antrags auf Ausstand der ersten Landschreiberin Frau Elisabeth Heer Dietrich begründet der zweite Landschreiber Herr Nic Kaufmann mittels der \ac{VDLK} (\ac{VDLK}) das Nichteintreten bzgl. des Ausstands.

\Rz Bei der Verfügung vom 27. Oktober 2022 muss es sich jedoch um einen Entscheid des Regierungsrats und gerade nicht der Landeskanzlei handeln. Ich habe daher die systematische Gesetzessammlung des Kantons Basel-Landschaft bzgl. des Regierungsrats durchgesehen und bin dabei auf die \ac{VGRR} aufmerksam geworden.\\

\Rz Im § 8 \ac{VGRR} wird das Unterzeichnen von Dokumenten des Regierungsrats geregelt. Gemäss § 8 Abs. 1 müssen u.a. Urkunden, und darunter fallen auch Verfügungen, vom Regierungsratspräsident und dem Landschreiber (handschriftlich) unterschrieben werden.\\

\Rz Im § 8 Abs. 2  \ac{VGRR} gilt, dass bei Dokumenten, die nicht spezifiziert sind, namentlich  Protokollauszüge, die der Landschreiber oder die Landschreiberin allein unterzeichnet.\\

\Rz Das \ac{BuGe} hat in seinem Entscheid BGE 131 V 483 (siehe \cite{BGE131V483}) in einem analogen Fall entschieden, dass dieser formale Fehler als so gravierend eingestuft werden muss, dass es in einem solchen Fall keine Heilung mittels einer Beschwerde bzgl. Einsprache-Entscheid nachträglichen Unterschrift zulässt und daher eine solche Verfügung ohne eine vorherigen materiellen Prüfung aufzuheben ist. Zusätzlich muss das zugrunde liegende Verfahren erneut von der Vorinstanz durchlaufen werden.


\chapter{Zulässige Beschwerdegründe nach BGG}

\section{Unrichtige Feststellung des Sachverhalts -- Art. 97 BGG} 
\Rz Die Feststellung des Sachverhalts kann nur gerügt werden, wenn sie offensichtlich unrichtig ist oder auf einer Verletzung des Bundesrechts und/oder einer Verletzung von kantonalen verfassungsmässigen Rechten beruht und wenn die Behebung des Mangels für den Ausgang des Verfahrens entscheidend sein kann.\\

\Rz Im Urteil wird auf die konkrete Aufführung meiner Beschwerde bzgl. der fehlenden Unterschrift des Regierungsratspräsidenten verzichtet. Eine Nennung dieser Beschwerde im rechtserheblichen Sachverhalt hätte bereits ohne eine materielle Prüfung zur Aufhebung der Verfügung vom 27. Oktober 2022 führen müssen. Damit liegt eine unrichtige Feststellung des Sachverhalts, die am Bundesgericht beschwert werden kann.\\

\Rz Im Urteil wird auf die konkrete Aufführung meiner Beschwerde bzgl. der Falschbeurkundung im Amt verzichtet. Eine Nennung dieser Beschwerde im rechtserheblichen Sachverhalt hätte bereits ohne eine materielle Prüfung zur Aufhebung der Verfügung vom 27. Oktober 2022 führen müssen. Damit liegt eine weitere unrichtige Feststellung des Sachverhalts vor, die selbstständig am Bundesgericht beschwert werden kann.



\section{Verletzung des rechtliches Gehörs}
\Rz Der in Art. 29 Abs. 2 BV verankerte Anspruch auf rechtliches Gehör bildet bereits Teilgehalt des allgemeinen Grundsatzes auf ein faires Verfahren (Art. 29 Abs. 1 BV). Der Anspruch auf rechtliches Gehör gilt für alle, die in einem Rechtsanwendungsverfahren Parteistellung haben, unabhängig von ihrer Berechtigung in der Sache. Er erfüllt im Wesentlichen auch dieselben individualrechtlichen und rechtsstaatlichen Funktionen wie das allgemeine Fairnessprinzip, dient aber darüber hinaus auch der Sachverhaltsaufklärung (BSK BV-WALDMANN, Art. 29 BV, N. 41).\\

\subsection{hinsichtlich rechtliches bzgl. der \acl{SFaP}} \label{RGFalsch}
\Rz Ich habe das \ac{KG} in meiner Beschwerde vom 07. Oktober 2022 unter E3 plus E3.1 darauf hingewiesen, dass die Bewilligung der \ac{SFaP} mittels einer Falschbeurkundung im Amt stattgefunden hat und die Bewilligung daher für mich nicht nutzbar ist.\\

\Rz Hätte das \ac{KG} mein Argument bzgl. der Falschbeurkundung im Amt gehört, dann hätte diese Begründung bereits ausreichen müssen um die Verfügung vom 27. Oktober 2022 aufzuheben und das Verfahren an die Vorinstanz zurückzuweisen. 
 
\subsection{hinsichtlich der fehlenden Unterschrift} \label{RGUnterschrift}

 In der Beschwerdeschrift vom 07. Oktober 2022 habe ich das \acl{KG} auf den § 8 Abs. 1 \ac{VGRR} hingewiesen. Ich habe das \ac{KG} auf die Formerfordernisse bei einer Verfügung des Regierungsrats hingewiesen. 
 einer Unterschrift der schriftlichen Eingabe auf dem Postweg offensichtlich bewusst gewesen. Im E-Mail vom 24. Oktober 2014 habe er erklärt, das Original der Einsprache sei auf dem Postweg unterwegs. Das am 30. Oktober 2014 der Schweizerischen Post übergebene, eigenhändig unterzeichnete Original der Einsprache sei erst nach Ablauf der Einsprachefrist am 31. Oktober 2014 und damit verspätet bei der Beschwerdegegnerin eingetroffen. Da der Beschwerdeführer darauf hingewiesen habe, das Original der Einsprache sei auf dem Postweg unterwegs, habe für die Beschwerdegegnerin kein Anlass bestanden, eine Nachfrist im Sinne von Art. 10 Abs. 5 ATSV zur Behebung eines Formmangels anzusetzen, zumal dieser zum damaligen Zeitpunkt nicht bekannt gewesen sei, dass der Versicherte nicht mehr durch seine Rechtsschutzversicherung beraten und vertreten gewesen sei.



\Rz Ich habe das \ac{KG} in meiner Beschwerde vom 07. Oktober 2022 unter E2 darauf hingewiesen, dass bei der Verfügung vom 27. Oktober 2022 die obligatorische Unterschrift des Regierungsratspräsidenten fehlt und das Bundesgericht bereits in einem früheren Verfahren entschieden hat, dass in einem solchen Fall die Verfügung aufgehoben werden muss.\\

\Rz Hätte das \ac{KG} mein Argument bzgl. der fehlenden Unterschrift des Regierungsratspräsidenten gehört, dann hätte diese Begründung bereits ausreichen müssen um die Verfügung vom 27. Oktober 2022 aufzuheben und das Verfahren an die Vorinstanz zurückzuweisen. 

\section{Verletzung des Rechtsverweigerungverbots i.e.S}
\Rz Auch ein Kantonsrichter hat sich nach geltenden Gesetzen zu richten und den rechtlich erheblichen Sachverhalt schriftlich festzustellen. Wenn der Richter ihm vorliegende rechts relevante Tatsache nicht schriftlich würdigt, dann liegt eine Rechtsverweigerung i.e.S. vor. Genau das ist im Urteil passiert, denn es wird weder das Argument der Urkundenfälschung noch die von mir bemängelte fehlenden Unterschrift vom Regierungsradpräsidenten schriftlich gewürdigt. 

Stattdessen wird mir das notwendige schutzwürdige Interesse für die Beschwerde abgesprochen.



\section{Unrichtige Feststellung des Sachverhalts}
\Rz Art. 97 Abs. 1 BGG umschreibt, unter welchen Voraussetzungen vor Bundesgericht ausnahmsweise die Rüge zulässig ist, dem angefochtenen Entscheid liege eine fehlerhafte Sachverhaltsfeststellung zugrunde. Sachverhaltsfragen beziehen sich auf die tatsächlichen Gegebenheiten. Mit der Feststellung des Sachverhalts beantwortet das Gericht demnach die folgenden Fragen: wer, wo,
wann, was, wie, warum? Auch Rechtshandlungen und insbesondere Prozesshandlungen gehören zum Sachverhalt (BSK BGG-SCHOTT, Art. 95 N. 1-59, N. 28; DONZALLAZ, Commentaire, Art. 95 N 3672; CORBOZ, SJ 2006, 340). Art. 97 Abs. 1 BGG umfasst zwei Varianten. Die Variante der offensichtlichen Unrichtigkeit ist erfüllt, wenn die Sachverhaltsfeststellung eindeutig und augenfällig unzutreffend ist. D.h. es muss ein klares Abweichen der tatsächlichen Gegebenheiten von der Sachverhaltsfeststellung im angefochtenen Entscheid zu bejahen sein. Wird der rechtlich relevante Sachverhalt in diesem Sinne offensichtlich unkorrekt ermittelt und wirkt sich dies auf den Entscheid aus so liegt auch eine willkürliche Rechtsanwendung (Art. 9 BV) vor. Gemäss der zweiten Variante kann eine mangelhafte Sachverhaltsermittlung auch gerügt werden, wenn diese auf einer Rechtsverletzung im Sinne von Art. 95 BGG beruht. Die Sachverhaltsermittlung kann gemäss beiden Varianten von Art. 97 Abs. 1 BGG nur auf Beschwerde hin überprüft werden, wenn die Behebung des Mangels für den Ausgang des Verfahrens entscheidend sein kann. D.h. die beschwerdeführende Partei muss dartun können, dass bei korrekter Ermittlung des Sachverhalts ein anderer Entscheid in der Sache möglich ist (BSK BGG-SCHOTT, Art. 97, N. 9 f.).

\subsection{hinsichtlich der \acl{SFaP}} \label{RGFalsch}
\Rz Ich habe das \ac{KG} in meiner Beschwerde vom 07. Oktober 2022 unter E3 plus E3.1 darauf hingewiesen, dass die Bewilligung der \ac{SFaP} mittels einer Falschbeurkundung im Amt stattgefunden hat und die Bewilligung daher für mich nicht nutzbar ist.\\

\Rz Hätte das \ac{KG} mein Argument bzgl. der Falschbeurkundung im Amt im rechtlich erheblichen Sachverhalt aufgenommen, dann hätte dies bereits ausreichen müssen um die Verfügung vom 27. Oktober 2022 aufzuheben und das Verfahren an die Vorinstanz zurückzuweisen. 


\subsection{hinsichtlich der fehlenden Unterschrift} 
\Rz Ich habe das \ac{KG} in meiner Beschwerde vom 07. Oktober 2022 unter E2 darauf hingewiesen, dass bei der Verfügung vom 27. Oktober 2022 die obligatorische Unterschrift des Regierungsratspräsidenten fehlt und das \ac{BGE} bereits in einem früheren Verfahren (siehe \cite{BGE121II473}) entschieden hat, dass in einem solchen Fall die Verfügung aufgehoben werden muss.\\

\Rz Hätte das \ac{KG} mein Argument bzgl. der fehlenden Unterschrift des Regierungsratspräsidenten im rechtlich erheblichen Sachverhalt aufgenommen, dann hätte dies bereits ausreichen müssen um die Verfügung vom 27. Oktober 2022 aufzuheben und das Verfahren an die Vorinstanz zurückzuweisen. 

\chapter{Rechtliches}

\section{Rechtliches bzgl. der \acl{SFaP}}

\Rz In Basel-Land gibt es grundsätzlich eine freie Schulwahl. Eltern können die private Schulung bei der \ac{BKSD} beantragen oder auf eigene Kosten an einer vom Kanton zugelassenen Privatschule beschulen lassen. Oder die Eltern wählen die öffentliche Schule für ihr Kind aus. Sollten die Eltern die öffentliche Schule gewählt haben, kommt eine weitere Option ins Spiel, die \acl{SFaP}.\\

Hinweis: Als «private Schulung» gilt die Beschulung der eigenen Kinder durch die Erziehungsberechtigten oder eine von ihnen mit der Beschulung beauftragte Drittperson in einer Gruppe von bis zu 8 Schülerinnen und Schülern. Auf die private Schulung wird in diesem Dokument nicht weiter eingegangen.



\subsection{Rechtsgrundlagen bzgl. Privatschulen} \label{rgPrivatSchule}
\Rz Die Rechtsgrundlagen zum Führen einer Privatschule ergeben sich aus dem \ac{GBiG} sowie \ac{VPuP}. Danach bedarf es im Kanton Basel-Landschaft für die Führung einer Privatschule einer Bewilligung der \ac{BKSD}. Die Bewilligung wird auf Antrag erteilt, wenn die an die öffentlichen Schulen gestellten Anforderungen erfüllt sind (siehe § 15 Abs. 1,2 \ac{GBiG}).  Dem Antrag ist u.a. das pädagogische Konzept, das aufzeigt, wie und in welcher Form die Schule die Schülerinnen und Schüler ausbildet (z.B. pädagogisches Vorgehen, Stundentafel, Schulfächer, Anzahl Lektionen, Unterrichtszeiten) beizulegen (siehe § 3 Abs. 2 \ac{VPuP}).\\ 

\Rz Der Kanton, genauer die das Amt für Volksschulen, Hauptabteilung Sonderpädagogik, schliesst Leistungsvereinbarungen u.a. mit Privatschulen bzgl. der \ac{SFaP} ab. Es können auch Privatschulen anderer Kantone zum Zuge kommen, wenn diese über eine Bewilligung als Privatschule des Standortkantons verfügen. (siehe § 55 \ac{VSoP}).


\subsection{Rechtsgrundlagen bzgl. der \ac{SFaP}} \label{rgSFaP}
\Rz Der kantonale Gesetzgeber  hat bezüglich einer Bewilligung der \ac{SFaP} im \ac{BiG} folgenden Rechtsrahmen festgelegt.\\  

\Rz Die \ac{BKSD} kann ein Angebot der \ac{SFaP} oder weiteren Leistungserbringenden im Bildungsbereich übertragen. Vorrang haben Massnahmen der Speziellen Förderung innerhalb der öffentlichen Schulen des Kantons und der Einwohnergemeinden. Die Bewilligung zur Aufnahme einer \ac{SFaP} oder bei einem weiteren Leistungserbringenden im Bildungsbereich erteilt die \ac{BKSD} auf Antrag einer vom Kanton bestimmten Fachstelle. Vorgängig der Erteilung einer Bewilligung zugunsten einer Schülerin oder eines Schülers des Kindergartens oder der Primarschule nimmt die \ac{BKSD} Rücksprache mit dem zuständigen Schulrat. (§ 46 \ac{GBiG})\\

\Rz Mit der Revision des \ac{BiG} aus 2020 hat der Gesetzgeber den Begriff des Vorrangs mittels des neuen § 5a Abs. 1 \ac{GBiG} präzisiert. Danach ist der Vorrang im § 46 \ac{GBiG} wie folgt zu verstehen (Zitat): \textit{Die Schülerinnen und Schüler mit besonderem Bildungsbedarf werden vorzugsweise integrativ geschult, unter Beachtung des Wohles und der Entwicklungsmöglichkeiten des Kindes oder des Jugendlichen sowie unter Berücksichtigung des schulischen Umfeldes und der Schulorganisation.}\\

\Rz Die auf der Stufe des Kindergartens oder der Primarschule geforderte Rücksprache mit dem zuständigen Schulrat ergibt sich aus dem § 13 Abs. 1,2 \ac{GBiG}. Danach ist die Einwohnergemeinden der (Kosten-)Träger des Kindergartens oder der Primarschule sowie der speziellen Förderung auf der Primarstufe. Der Schulrat wurde aus zwei Gründen bestimmt. Erstens gehört dem Schulrat ein Mitglied des Gemeinderats an und zweitens genehmigt der Schulrat das Schulprogramm und kennt damit das pädagogische und organisatorische Konzept der Schule.\\

 


 dass Massnahmen der Speziellen Förderung innerhalb der öffentlichen Schulen des Kantons und der Einwohnergemeinden Vorrang geben werden (§ 46 Abs. 1). unter § 46 Abs. 2 obligatorisch und unmissverständlich, dass ein Antrag einer vom Kanton bestimmten Fachabteilung vorliegen muss.\\




\Rz Der Regierungsrat Basel-Landschaft hat auf dieser Basis in der \ac{VSPD} mit dem § 2 Abs. 4 verfügt, dass der \ac{SPD} seine Bemühungen in den Dienst positiver Schullaufbahnen stellt und mit Zustimmung der Inhaber der elterlichen Sorge bei den zuständigen Behörden die notwendigen Massnahmen zu beantragen hat.\\

\Rz Gemäss § 25 des \ac{VwVGBL}  führt eine Behörde das Verfahren auf Erlass einer Verfügung auf Begehren oder von Amtes wegen durch. Dem Begehren um Erlass einer Verfügung ist zu entsprechen, wenn ein schutzwürdiges Interesse nachgewiesen wird. Fehlt ein schutzwürdiges Interesse, so tritt die Behörde auf das Begehren nicht ein.\\

\Rz Das schutzwürdige Interesse wird vom \ac{SPD} festgestellt, in dem dieser die zu beratenen Personen in Fragen des Lernens, des Verhaltens und der Entwicklung unterstützt und die notwendigen Massnahmen für eine  positive Schullaufbahn mittels der dazu notwendigen psychologischen Abklärungen und Methoden ermittelt.

\subsection{Rechtsauffassung in der Verwaltung bzgl. der \ac{SFaP}} \label{Rechtsauffasung}
\Rz Die Rechtsauffassung  in der Verwaltung bzgl. der \acl{SFaP} (\ac{SFaP}) lässt sich am besten anhand der Rechtshinweise in der \acl{eEI} (\ac{eEI}) sowie dem \acl{eAE} (\ac{eAE}) erfassen. In beiden Dokumenten befindet jeweils am Ende folgender identische rechtlicher Hinweis:\\
\label{rechtshinweis}\Rz
\begin{addmargin}[2.5em]{0em}\emph{Gestützt auf den Antrag der Erziehungsberechtigten, die Empfehlung der kantonalen Abklärungsstelle (SPD/KJP) und die Stellungnahme der Schulleitung der Regelschule entscheidet das Amt für Volksschulen, Abteilung Sonderpädagogik, über die Massnahmen der Speziellen Förderung nach § 46 des Bildungsgesetzes. Der Entscheid wird den Erziehungsberechtigten, der Schulleitung der Regelschule und der Abklärungsstelle schriftlich zugestellt. Auf der Primarstufe muss eine Kostengutsprache der Gemeinde vorliegen.\\
}\end{addmargin}    

\Rz Gemäss der Rechtsauffassung innerhalb der Verwaltung, darf danach eine Bewilligung der \ac{SFaP} auch Basis eines Antrags der Eltern und einer Empfehlung des \ac{SPD} erteilt werden. Diese Rechtsauffassung impliziert indirekt, dass eine Empfehlung des \ac{SPD} als formeller Antrag des \ac{SPD} gelten darf. Nur auf Annahme dieser Bedingung kann die Verwaltung ein willkürliches Verwaltungshandeln negieren. 

\subsection{Können Eltern im Antrag die \ac{SFaP} rechtlich sinnvoll begründen?}
\Rz Um die Frage beantworten zu können, muss den § 46 Abs. 1 \ac{BiG} in die Überlegung einbeziehen. Danach haben (gleichartige) Massnahmen der Speziellen Förderung innerhalb der öffentlichen Schulen des Kantons und der Einwohnergemeinden den Vorrang eingeräumt zu werden.\\

\Rz Damit die Eltern im Antrag die \ac{SFaP} rechtlich sinnvoll begründen können, müssten diese alle (gleichartige) Massnahmen der Speziellen Förderung innerhalb der öffentlichen Schulen des Kantons und der Einwohnergemeinden kennen. Bereit hier wird deutlich, dass dies von Eltern im Allgemeinen nicht zu leisten ist.\\ 

\Rz Aus obiger Überlegung  wird deutlich, dass Eltern im Allgemeinen im Antrag die \ac{SFaP} nicht rechtlich sinnvoll begründen können. Zusätzlich müssten die Eltern zuvor Kenntnis davon erlangen, dass es die Möglichkeit eine \ac{SFaP} überhaupt gibt.\\

\Rz Um die Möglichkeiten einer Unterstützung bzw. Fördermöglichkeit für ihre Kinder herauszufinden gehen die Eltern gerade ratsuchend auf den \acl{SPD} zu. 

\subsection{Kann eine Empfehlung des \ac{SPD} als formeller Antrag gelten?}
\Rz Unter Berücksichtigung des § 15 Abs. 1 des \acl{VwVGBL} (\ac{VwVGBL}) kann eine mündliche Empfehlung des \ac{SPD} bereits nicht als formeller Antrag gelten, denn Anträge (rechtliche Begehren) bedürfen obligatorisch der Schriftform sowie einer (handschriftlichen) Unterschrift.\\

\Rz Stellt sich nun noch die Frage, ob eine schriftliche Empfehlung als formeller Antrag gelten kann, wenn diese ein klar umschriebenes Begehren, die Angabe der Tatsachen und Beweismittel, eine Begründung sowie die Unterschrift des zuständigen Schulpsychologen beinhaltet. Um diese Frage beantworten zu können muss eines bzgl. des 
\ac{VwVGBL} beachtet werden. Grundsätzlich ist das Herantreten der Eltern an den \acl{SPD} als Anfrage anzusehen. Der \ac{SPD} beschäftigt sich mir der Anfrage und beantwortet diese mittels einer Empfehlung (Indikation).\\

\Rz Der Regierungsrat Basel-Landschaft hat die Beantwortung dieser Fragestellung mehr oder weniger bereit, wenn auch indirekt beantwortet, verfügt dieser doch im § 2 Abs. 4 der \acl{VSPD}, dass der \acl{SPD} die vom \ac{SPD} notwendige erachteten Massnahmen bei der dafür zuständigen Behörde zu beantragen hat. Aufgrund der Tatsache, dass die \ac{VSPD} auf den Aussenbereich des \ac{SPD} zielt, kommt der Verordnung  ein gesetzlicher Charakter zu. Womit der \ac{SPD} per Amts wegen einen Antrag bzgl. der \acl{SFaP} zu stellen hat. Diese Bestimmung wäre nicht notwendig, wenn bereits eine Empfehlung (Indikation) den Charakter eines rechtlichen Begehrens innehätte.\\

\Rz Die zu klären Frage, ob eine Empfehlung des \ac{SPD} als formeller Antrag verwendet werden darf, muss daher im Allgemeinen mit Nein beantwortet werden.

\subsection{Fazit Rechtsauffassung in der Verwaltung bzgl. der \ac{SFaP}} \label{FazitSFaP}   
\Rz Die in der Verwaltung bestehenden und \ac{gBP} bzgl. der \ac{SFaP} führt bei Anwendung zu einer Falschbeurkundung im Amt. Eine vom \ac{SPD} Empfehlung (Indikation), egal ob mündlicher oder schriftlicher Natur, hat nicht den gesetzlichen Charakter eines rechtlichen Begehrens. Eltern sind keine vom Kanton bestimmte Fachstelle und sind somit bzgl. der \ac{SFaP} nicht antragsberechtigt.\\

\Rz Eine Bewilligung der \acl{SFaP} darf nur ausgestellt werden, wenn alle obligatorisch notwendigen Tatsachen erfüllt sind. Wenn die Verwaltung als auf Basis eines Antrags der Eltern und einer schriftlichen Empfehlung (Indikation) eine \ac{SFaP} bewilligen, wird damit indirekt zum Ausdruck gebracht, dass aufgrund des Vorliegens dieser beiden Schriftstücke eine Bewilligung rechtsstaatlich erlaubt ist. Dass dem nicht so ist, habe ich bereits im Abschnitt \ref{rechtsgrundlage} aufgezeigt. Somit wird bei einer Bewilligung auf Grundlage eines Antrags der Eltern und einer schriftlichen Empfehlung (Indikation) eine rechtlich erhebliche Tatsache unwahr beurkundet, was gerade den Wesenszug einer Falschbeurkundung im Amt auszeichnet.\\

\Rz In diesem Abschnitt wurde somit gleichzeitig nach gewiesen, dass die \ac{gBP} aus strafrechtlichen Gründen zu beanstanden ist.


\chapter{Kosten}
Dem Ausgang des Verfahrens entsprechend sind die ordentlichen sowie ausserordentlichen Kosten der Beschwerdegegnerin 1 und dem Beschwerdegegner 2 aufzuerlegen. 

\newpage

\chapter{Abkürzungsverzeichnis}
\begin{acronym}
	\acro{AVS}[AVS]{Amt für Volksschulen}
	\acro{BGE}[BGE]{Entscheid des schweizerischen Bundesgericht}
	\acro{BiG}[BiG]{Bildungsgesetz Basel-Landschaft}
	\acro{BKSD}[BKSD]{Bildungs-, Kultur- und Sportdirektion}
	\acro{BuGe}[BuGe]{Schweizerisches Bundesgericht}
	\acro{BZB}[BZB]{Bildungszentrum Basel}
	\acro{eEI}[Erstempfehlung]{Empfehlung (Indikation) vom 08. Januar 2022 (siehe \cite{Erstindikation})}
	\acro{eAE}[Erstantrag]{Antrag der Eltern vom 10. Februar 2022 (siehe \cite{Elternantrag})}
	\acro{gBP}[gBP]{gelebte Bewilligungspraxis bzgl. der spezielle Förderung an Privatschule}
	\acro{KG}[KaGe]{Kantonsgericht Basel-Landschaft, Abteilung Verfassungs- und Verwaltungsrecht}
	\acro{KGE}[KGE]{Entscheid des Kantonsgericht Basel-Landschaft}
	\acro{SPD}[SPD]{Schulpsychologogischer Dienst}
	\acro{SFaP}[SFaP]{Spezielle Förderung an Privatschule}
\end{acronym}

\chapter{Verzeichnis Gesetze \& Verordnungen }
\begin{acronym}
	\acro{VDLK}[SGS 147.11]{Dienstordnung der Landeskanzlei}
	\acro{VGRR}[SGS 141.11]{Geschäftsordnung des Regierungsrates}
	\acro{VwVGBL}[SGS 175]{Verwaltungsverfahrensgesetz Basel-Landschaft (VwVG BL)}
	\acro{GVPOBL}[SGS 271]{Gesetz über die Verfassungs- und Verwaltungsprozessordnung}
	\acro{GBiG}[SGS 640]{Bildungsgesetz Basel-Landschaft}
	\acro{VPuP}[SGS 640.43]{Verordnung über die Privatschulen und die private Schulung}
	\acro{VSoP}[SGS 640.71]{Verordnung Sonderpädagogik} 
	\acro{VSPD}[SGS 645.21]{Verordnung über den Schulpsychologischen Dienst}
\end{acronym}



\newpage

\printbibliography[keyword={obligatorisch},title={\underline{Beweismittel -- obligatorisch}}]
\printbibliography[keyword={nachweis},title={\underline{Beweismittel -- bzgl. meiner Aussagen}}]
\printbibliography[keyword={AkteKG},title={\underline{Beweismittel -- aus Verfahrensakte}}]
\printbibliography[keyword={KGE},title={\underline{Beweismittel -- \ac{KGE} (online)}}]
\printbibliography[keyword={BGE},title={\underline{Beweismittel -- \ac{BGE} (online)}}]

\newpage
\includepdf{BeschwerdeAnhang}


\end{document}